\section{Esercizio Algoritmo del massimo}
\begin{flushleft}

Facciamo un analisi uguale a quella vista prima ma con un algoritmo di ricerca del massimo 
in un vettore.

$\newline$

Vediamo lo pseudocodice dell'algoritmo:

\begin{lstlisting}
Min(A) {
	min <- A[1]
	
	for (i <- 2 to A.length) {
		if (min > A[i]) {
			min <- A[i]
		}
	}

}
\end{lstlisting}
\end{flushleft}

\subsection{Complessità}

\begin{flushleft}
Recuperiamo la funzione che ci ritornerà il tempo usato dall'algoritmo:

$\newline$

\codewords{min <- A[1]}

$c_1$

$\newline$

\codewords{for (i <- 2 to A.length)}

$c_2 + c_3 (n-1)$

\codewords{if (min > A[i])}

$c_4 (n - 1)$

\codewords{min <- A[i]}

$c_5 (n-1)$

Quindi abbiamo che:

\[T_{Max}(n) = c_1 + c_2 + c_3 (n-1) + c_4 (n-1) + c_5 (n-1)\]

che possiamo semplificare in:

\[T_{Max}(n) = a + b (n-1) = \Theta(n)\]

In questo caso non abbiamo fatto un'analisi del caso peggiore e del caso migliore poiché
la complessità temporale non cambia.

\end{flushleft}

\subsection{Correttezza}

\begin{flushleft}

Per dimostrare la correttezza dobbiamo dimostrare l'invariante del ciclo for che nel
nostro caso è quello che all'$i$-esima iterazione, \codewords{max} è uguale al valore massimo
del vettore  

\end{flushleft}