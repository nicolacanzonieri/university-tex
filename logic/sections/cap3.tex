\section{Capitolo  3: Forma normale congiuntiva e disgiuntiva}
\begin{flushleft}
L'obbiettivo di questo capitolo è quello di trasformare una formula proposizionale
in una formula ad essa logicamente equivalente che abbia una forma particolarmente
semplice dal punto di vista sintattico che ci consentirà di valutare in maniera più
semplice la sua validità e la sua soddisfacibilità.
\end{flushleft}

\subsection{Definizione di forma normale congiuntiva e disgiuntiva}

\begin{flushleft}
Un \textbf{letterale} è una lettera proposizionale oppure la negazione di una 
lettera proposizionale.

Se $p$ è una lettera proposizionale, $\lbrace p, \neg{p} \rbrace$ è una
\textbf{coppia complementare di letterali}. Più in generale se $F$ è una formula, 
$\lbrace F, \neg{F} \rbrace$ è una \textbf{coppia complementare}. Diciamo inoltre
che $F$ e $\neg{F}$ sono uno il \textbf{complemento} dell'altro.

\bigskip

\textbf{Lemma 3.3}

Un insieme di letterali è soddisfacibile se e solo se non contiene nessuna coppia
complementare.

\bigskip

\textbf{Esercizio 3.5}

Vogliamo dimostrare che una disgiunzione di letterali è valida se e solo se tra 
i disgiunti vi è una coppia complementare.

\medskip

Dimostriamo che $p \lor q \text{ valida } \Rightarrow \lbrace p, q \rbrace \text{ è 
una coppia complementare}$.

\smallskip

Per ipotesi abbiamo che $p \lor q$ è valida che vuol dire che per ogni interpretazione 
possibile abbiamo che la formula $F = p \lor q$ è soddisfatta, cioè 
$\forall v, \: v(F) = \mathbb{V}$. Dobbiamo dimostrare che $\lbrace p, q \rbrace$ è 
una coppia complementare, cioè $q = \neg p$.

Poiché $F = p \lor q$ è una disgiunzione, ed soddisfatta, $\forall v$ abbiamo che 
se $v(p) = \mathbb{F}$ allora $v(q) = \mathbb{V}$ oppure il viceversa. Tuttavia ricordiamo
che esistono due casi importanti nella disgiunzione e sono i casi in cui $v(p) = \mathbb{V},
v(q) = \mathbb{V}$ e $v(p) = \mathbb{F}, v(q) = \mathbb{F}$. Nel primo caso si avrebbe che $F$
è comunque soddisfatta ma nel secondo caso no. Questa è la prova che automaticamente
implica che ci sia una relazione di complementarità tra $p$ e $q$, infatti solo se 
$q = \neg p$ (oppure $p = \neg q$) si avrà che $F$ è soddisfatta per qualsiasi caso.

\medskip

Dimostriamo ora che $\lbrace p, \neg p \rbrace \Rightarrow p \lor \neg p$ è valida.

\begin{proof}[\unskip\nopunct]
Ma questo è ovvio anche in parte per quanto visto prima
\end{proof}

\bigskip

Diciamo che una formula proposizionale è in \textbf{forma normale congiuntiva} se
è della forma $F_1 \land F_2 \land ... \land F_m$ dove per $1 \le i \le m$, $F_i$
è della forma $G_{i,1} \lor G_{i,2} \lor ... \lor G_{i,h_i}$ dove per 
$1 \le j \le h_i$, $G_{i, j}$ è un letterale.

\bigskip

Vediamo di fare qualche esempio:

\textbf{Esempio:}

\[(p \lor q) \land (r \lor s)\]

\end{flushleft}