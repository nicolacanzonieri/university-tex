\section{Capitolo 2: Probabilità classica, frequentista e assiomatica}

\subsection{Operazioni su eventi}

\begin{flushleft}

Abbiamo quattro principali operazioni logiche su eventi:

\begin{itemize}
\item{Evento contrario}
\item{Unione}
\item{Intersezione}
\item{Differenza}
\end{itemize}

\textbf{Evento contrario:}

È quell'evento $\overline{A}$ che si verifica quando non si verifica $A$

$\newline$

\textbf{Unione:}

È quell'evento $A \cup B$ che si verifica quando si verifica $A$ oppure $B$ (oppure entrambi)

$\newline$

\textbf{Intersezione:}

È quell'evento $A \cap B$ che si verifica quando si verificano entrambi $A$ e $B$

$\newline$

\textbf{Differenza:}

È quell'evento $A \setminus B$ che si verifica quando si verifica $A$ ma non $B$

$\newline$

Valgono inoltre le leggi di \textbf{De Morgan}:

\[ \overline{(A \cup B)} = \overline{A} \cap \overline{B}
\text{ e }
\overline{(A \cap B)} = \overline{A} \cup \overline{B} \]

$\newline$

Diciamo inoltre che se due eventi non possono realizzarsi simultaneamente allora sono
due \textbf{eventi incompatibili}. Tramite formule abbiamo che questo è vero se:

\[ A \cup B = \emptyset \]

\end{flushleft}

\subsection{Classe degli eventi}

\begin{flushleft}

La \textbf{classe degli eventi} è la collezione di tutti gli eventi di cui ha senso parlare 
nel contesto del dato esperimento casuale con spazio campionario $S$. È indicata con 
$\mathcal{A}$.

$\newline$

Adesso che abbiamo lo spazio campionario $S$ e la classe degli eventi $\mathcal{A}$ possiamo
modellare l'incertezza presente nell'esperimento casuale $\varepsilon$.

Per ogni $E \in \mathcal{A}$ indichiamo con $P(E)$ la \textbf{probabilità} di $E$, che sarà
pertanto un'applicazione $P : \mathcal{A} \rightarrow \mathbb{R}$. Per la costruzione effettiva
della probabilità ci sono bisogno di alcuni schemi concettuali, dove la \textbf{probabilità
classica}, \textbf{frequentista} e \textbf{assiomatica} sono i più
importanti.

\end{flushleft}

\subsection{Probabilità Classica}

\begin{flushleft}

Sia $\varepsilon$ un esperimento casuale, $S$ il suo spazio campionario con cardinalità
finita $|S|$ e la classe degli eventi $\mathcal{A} = \mathcal{P}(S)$, pure con cardinalità
finita $2^{ |S| }$.

Allora il modo più semplice di valutare $P(E)$ è secondo la definizione classica di Laplace:

\[ P(E) = \frac{|E|}{|S|}\]

Notiamo che questa definizione suppone che tutti i risultati siano egualmente possibili.

\end{flushleft}

\subsection{Probabilità frequentista}

\begin{flushleft}

Dati $\varepsilon$, $S$, $\mathcal{A}$ e un evento $E \in \mathcal{A}$ nella concezione
frequentista $P(E)$ misura la propensione dell'esperimento $\varepsilon$ a produrre la 
realizzazione di $E$ in base alla sue preassegnate regole.

Assumendo omogeneità nel tempo delle ripetizioni dell'esperimento, il valore $P(E)$,
relativo a ciò che si deve ancora sperimentare, può venire empiricamente approssimato da 
un dato di esperienza. Si supponga quindi di replicare $\varepsilon$ un numero grande di volte
che indichiamo con $R$ e di prendere nota che l'evento si sia realizzato o meno nelle varie
replicazioni. Il valore $P(E)$ varrà quindi:

\[ P(E) \approx P_{R}(E) = \frac 1 R \sum_{r = 1}^R y_r \]

dove $y_r = 1$ se la replicazione r-esima si è verificata, mentre $y_r = 0$ altrimenti.

\end{flushleft}

\subsection{Probabilità assiomatica}

\begin{flushleft}

L'assiomatizzazione del Calcolo delle Probabilità offre le regole per calcolare 
correttamente nuove probabilità a partire da probabilità assegnate. Seguendo le regole, i
calcoli funzionano qualunque sia lo scopo dell'analisi e qualunque sia la nostra concezione
di probabilità.

Il primo passo per l'assiomatizzazione è descrivere bene la classe degli eventi 
$A \subseteq \mathcal{P}(S)$, cioè le operazioni sugli eventi non devono portare al di fuori
della classe degli eventi stessa.

In pratica la scelta di $A \subseteq \mathcal{P}(S)$ è guidata dalla cardinalità di $S$:

\begin{itemize}
\item{Se $S$ ha cardinalità finita o numerabile, si farà sempre la scelta 
$\mathcal{A} = \mathcal{P}(S)$}

\item{Se $S$ ha la cardinalità del continuo, $\mathcal{P}(S)$ risulta troppo grande e quindi
si deve scegliere una $A \subset \mathcal{P}(S)$ somigliante a $\mathcal{P}(S)$, cioè con 
analoghe proprietà di chiusura alle operazioni su eventi. I dettagli sono un po' tecnici}
\end{itemize}

\end{flushleft}

\subsection{La classe degli eventi come $\sigma$-algebra}

\begin{flushleft}

Si postula che $\mathcal{A}$ sia sempre una $\sigma$-algebra di parti di S, ossia una 
collezione di parti di $S$ tale che valgano le seguenti tre proprietà:

\begin{itemize}
\item{$i) \; S \in \mathcal{A}$}
\item{$ii) \; A \in \mathcal{A} \Rightarrow \overline{A} \in \mathcal{A}$}
\item{$iii) \; A_i \in \mathcal{A} \; \forall i \in I \subseteq \mathbb{N} \Rightarrow 
\bigcup_{i \in I} A_i \in \mathcal{A}$}
\end{itemize}

In sintesi $i)$ richiede che $S$ sia un evento e $ii)$ postula che se $A$ è un evento allora 
anche $\overline{A}$ sia sempre un evento. Il punto $iii)$ esige che, data una successione 
finita o numerabile di eventi, anche la loro unione sia un evento, ossia che si possa sempre
parlare della realizzazione di almeno uno degli eventi di una successione finita o numerabile.

Notiamo inoltre che anche $\emptyset$ è un evento, infatti $\overline{S} \in A$ per $i)$ e 
$ii)$.

\end{flushleft}

\subsection{Spazi probabilizzabili e Spazi probabilizzati}

\begin{flushleft}

La coppia ordinata $(S, \mathcal{A})$, dove $S$ è non vuoto e $\mathcal{A}$ è una 
$\sigma$-algebra di parti di $S$, è detta \textbf{spazio probabilizzabile}

$\newline$

Chiamiamo invece \textbf{misura di probabilità} l'applicazione:

\[ P : \mathcal{A} \rightarrow [0, 1] \]

che soddisfa le tre proprietà, dette \textbf{assiomi di Kolmogorov}:

\textbf{A1: } $P(A) \ge 0 \; \forall A \in \mathcal{A}$ detto assioma di non-negatività 

\textbf{A2: } $P(S) = 1$ detto assioma di normalizzazione

\textbf{A3: } per ogni successione $A_i, \; i \in I \subseteq \mathbb{N}$, di eventi a due
a due incompatibili vale:

\[ P(\cup_{i \in I} A_i) = \sum_{i \in I}P(A_i)\]

$\newline$

Avendo definito la \textbf{misura di probabilità} possiamo definire la terna ordinata 
$(S, \mathcal{A}, P)$ (dove $P$ è appunto la misura di probabilità) come 
\textbf{spazio probabilizzato}.

\end{flushleft}